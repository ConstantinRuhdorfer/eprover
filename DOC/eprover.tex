%%%%%%%%%%%%%%%%%%%%%%%%%%%%%%%%%%%%%%%%%%%%%%%%%%%%%%%%%%%%%%%%%%%%%%%%
%
% File  : eprover.tex
%
% Author: Stephan Schulz
%
% Contents
% 
%   Short introduction to some concepts of E.
% 
% Changes
%
% <1> Sat Dec 26 11:57:02 MET 1998
%     New
%%%%%%%%%%%%%%%%%%%%%%%%%%%%%%%%%%%%%%%%%%%%%%%%%%%%%%%%%%%%%%%%%%%%%%%%

\documentclass{article}

\usepackage{makeidx}
\usepackage{estyle}
\usepackage{supertabular}
\usepackage{url}

\author{Stephan Schulz}
\title{E 1.4\\[1.5ex]User Manual\\[1.5ex]{\normalsize --preliminary
    version--}}

\makeindex{}

\begin{document}
\maketitle{}

\begin{abstract}
  E is an equational theorem prover for full first-order logic, based
  on superposition and rewriting. In this \emph{very preliminary}
  manual we first give a short introduction for impatient new users,
  and then cover calculus, control, options and input/output of the
  prover in some more detail.
\end{abstract}

\tableofcontents{}

\section{Introduction}
\label{sec:intro}

This is a short and currently very sketchy documentation to the E
equational theorem prover. E is an purely equational theorem prover
for first-order logic with equality. It is based on paramodulation and
rewriting. This means that E reads a set of formulas and/or clauses
and saturates it by systematically applying a number of inference
rules until either all possible inferences have been performed or
until the empty clause has been derived, i.e. the clause set has been
found to be unsatisfiable and thus the conjecture has been proved.

E is still a moving target, but most recent releases have been quite
stable, and the prover is being used productively by several
independent groups of people. This manual should enable you to
experiment with the prover and to use some of its more advanced
features.

The manual assumes a working knowledge of refutational theorem
proving, which can be gained from e.g.~\cite{CL73}. For a short
description of E including performance data,
see~\cite{Schulz:IJCAR-2004}. A more detailed description has been
published as~\cite{Schulz:AICOM-2002}. Most papers on E and much more
information is available at or a few hops away from the E home page,
\url{http://www.eprover.org}.

Some other provers have influenced the design of E and may be
referenced in the course of this manual. These include
SETHEO~\cite{MILSGSM:JAR-97}, Otter~\cite{Mc94,MW:JAR-97},
SPASS~\cite{WGR96,WABCEKTT:CADE-99}, DISCOUNT~\cite{DKS97},
Waldmeister~\cite{BHF96,HJL:CADE-99} and
Vampire~\cite{RV:AICOM-2002,RV:IJCAR-2001}.


\section{Getting Started}
\label{sec:start}

Installation of E should be straightforward. The file \texttt{README}
in the main directory of the distribution contains the necessary
information. After building, you will find the stand-alone executable
\texttt{E/PROVER/eprover}.

E is controlled by a very wide range of parameters. However, if you do
not want to bother with the details, you can leave configuration for a
problem to the prover. To use this feature, use the following command
line options:

\bigskip
\noindent
\begin{tabular}{lp{7.5cm}}
  \texttt{-xAuto} & Choose a literal selection strategy and a
  clause evaluation heuristic automagically\index{xyzzy} (based on
  problem features).\\
  \texttt{-tAuto} & Select a term ordering automagically.\\
  \texttt{--memory-limit=xx} & Tell the prover how much memory
  (measured in MB) to use at most. In automatic mode E will optimize
  its behaviour for this amount (20~MB will work, 96~MB is
  reasonable, 768~MB is what I use. \emph{More is
  better}\footnotemark, but if you go over
  your physical memory, you will probably experience \emph{very} heavy
  swapping.).\\
\end{tabular}
\footnotetext{Emphasis added for E~0.7 and up, which globally cache
  rewrite steps.}
\begin{example}
  If you happen to have a workstation with 64 MB RAM\footnote{Yes,
    this is outdated. If it still applies to you, get a new
    computer! It will still work ok, though.}, the following command
  is reasonable:  
\small
\begin{verbatim}
eprover -xAuto -tAuto --memory-limit=48 PUZ031-1+rm_eq_rstfp.lop
\end{verbatim}
  \normalsize
\end{example}

This documentation will probably lag behind the development of the
latest version of the prover for quite some time. To find out more
about the options available, type \texttt{eprover --help} (or consult
the source code included with the distribution).



\section{Calculus and Proof Procedure}
\label{sec:calculus}

E is a purely equational theorem prover, based on ordered
paramodulation and rewriting. As such, it implements an instance of
the superposition calculus described in~\cite{BG94}. We have extended
the calculus with some stronger contraction rules and a more general
approach to literal selection. The proof procedure is a variant of the
\emph{given-clause} algorithm.


\subsection{Calculus}

\emph{Term}$(F,V)$\index{terms} denotes the set of (first order)
\emph{terms} over a finite set of function symbols \emph{F} (with
associated arities) and an enumerable set of variables
\emph{V}\index{variables}. We write $t|_p$ to denote the
subterm\index{subterm} of $t$ at a position $p$ and write
$t[p\leftarrow t']$ to denote $t$ with $t|_p$ replaced by $t'$. An
equation\index{equations} \eqn{s}{t} is an (implicitly symmetrical)
pair of terms. A positive literal\index{literal} is an equation
\eqn{s}{t}, a negative literal is a negated equation \neqn{s}{t}.  We
write \ueqn{s}{t} to denote an arbitrary
literal\footnote{Non-equational literals are encoded as equations or
  disequations \ueqn{P(t_1, \ldots, t_n)}{\top}. In this case, we
  treat predicate symbols as special function symbols that can only
  occur at the top-most positions and demand that atoms (terms formed
  with a top predicate symbol) cannot be unified with a first-order
  variable from $V$, i.e.\ we treat normal terms and predicate terms
  as two disjoint sorts. We sometimes write the literal $\eqn{P(t_1,
    \ldots, t_n)}{\top}$ as $P(t_1, \ldots, t_n)$ and $\neqn{P(t_1,
    \ldots, t_n)}{\top}$ as $\neg P(t_1, \ldots, t_n)$ for
  simplicity.} Literals can be represented as multi-sets of multi-sets
of terms, with \eqn{s}{t} represented as $\{\{s\},\{t\}\}$ and
\neqn{s}{t} represented as $\{\{s, t\}\}$.  A \emph{ground reduction
  ordering}\index{ordering}\index{reduction ordering} $>$ is a
Noetherian partial ordering that is stable w.r.t. the term structure
and substitutions and total on ground terms. $>$ can be extended to an
ordering $>_l$ on literals by comparing the multi-set representation
of literals with $>\!\!> >\!\!>$ (the multi-set-multi-set extension of
$>$).

Clauses are multi-sets of literals. They are usually represented as
disjunctions of literals, $\ueqn{s_1}{t_1} \vee \ueqn{s_2}{t_2} \ldots
\vee \ueqn{s_n}{t_n}$. We write \mw{Clauses(F,P,V)} to denote the set
of all clauses with function symbols $F$, predicate symbols $P$ and
variable $V$. If $\mathcal{C}$ is a clause, we denote the (multi-)set
of positive literals in $\mathcal{C}$ by $\mathcal{C^+}$ and the
(multi-)set of negative literals in $\mathcal{C}$ by $\mathcal{C^-}$

The introduction of an extended notion of \emph{literal selection} has
improved the performance of E significantly. The necessary concepts
are explained in the following.

\begin{definition}[Selection functions]
  \label{def:basics:inferences:selection}
  \index{selection functions}
  $\mw{sel}:\mw{Clauses(F,P,V)} \rightarrow \mw{Clauses(F,P,V)}$ is a
  \emph{selection function}, if it has the following properties for
  all clauses $\mathcal{C}$:
    \begin{itemize}
    \item $\mw{sel}(\mathcal{C}) \subseteq \mathcal{C}$.
    \item If $\mw{sel}(\mathcal{C})\cap \mathcal{C^-} = \emptyset$, then
      $\mw{sel}(\mathcal{C}) = \emptyset$.
    \end{itemize}
    We say that a literal $\mathcal{L}$ is \emph{selected} (with
    respect to a given selection function) in a clause $\mathcal{C}$
    if $\mathcal{L} \in \mw{sel}(\mathcal{C})$.
  \hfill$\blacktriangleleft$
\end{definition}

We will use two kinds of restrictions on deducing new clauses: One
induced by ordering constraints and the other by selection functions.
We combine these in the notion of \emph{eligible literals}.

\begin{definition}[Eligible literals]
  \index{literals!eligible}
  \index{eligible for paramodulation|see{literals, eligible}}
  \index{eligible for resolution|see{literals, eligible}}
  \index{eligible literals|see{literals, eligible}}
 \label{def:basics:inferences:eligible}
 Let $\mathcal{C} = \mathcal{L} \vee \mathcal{R}$ be a clause, let
 $\sigma$ be a substitution and let $\mw{sel}$ be a selection
 function. 
 \begin{itemize}
 \item We say $\sigma(\mathcal{L})$ is \emph{eligible for
     resolution} if either
   \begin{itemize}
   \item $\mw{sel}(C) = \emptyset$ and $\sigma(\mathcal{L})$ is
     $>_L$-maximal in $\sigma(\mathcal{C})$ or
   \item $\mw{sel}(C) \not= \emptyset$ and $\sigma(\mathcal{L})$ is
     $>_L$-maximal in $\sigma(\mw{sel}\mathcal{C})\cap \mathcal{C^-})$
     or
   \item $\mw{sel}(C) \not= \emptyset$ and $\sigma(\mathcal{L})$ is
     $>_L$-maximal in $\sigma(\mw{sel}(\mathcal{C})\cap
     \mathcal{C^+}))$.
   \end{itemize} 
 \item $\sigma(\mathcal{L})$ is \emph{eligible for paramodulation} if
   $\mathcal{L}$ is positive, $\mw{sel}(C) = \emptyset$ and
   $\sigma(\mathcal{L})$ is strictly $>_L$-maximal in
   $\sigma(\mathcal{C})$.
 \end{itemize}
 \hfill$\blacktriangleleft$
 \index{literals!eligible}
\end{definition}


The calculus is represented in the form of inference rules. For
convenience, we distinguish two types of inference rules. For
\emph{generating} inference rules, written with a single line
separating preconditions and results, the result is added to the set
of all clauses. For \emph{contracting} inference rules, written with a
double line, the result clauses are substituted for the clauses in the
precondition. In the following, $u$, $v$, $s$ and $t$ are terms,
$\sigma$ is a substitution and $R$, $S$ and $T$ are (partial) clauses.
$p$ is a position in a term and $\lambda$ is the empty or
top-position. $D \subseteq F$ is a set of unused constant predicate
symbols. Different clauses are assumed to not share any common
variables.


\begin{definition}[The inference system \textbf{SP}]
 \label{def:basics:inferences:sp}\index{SP@\textbf{SP} (calculus)}
 Let $>$ be a total simplification ordering (extended to orderings
 $>_L$ and $>_C$ on literals and clauses) and let $\mw{sel}$ be a
 selection function. The inference system \textbf{SP} consists of the
 following inference rules:
 
 \begin{itemize}
 \item \emph{Equality Resolution}\index{equality resolution}:
   
   \bigskip (ER) \GInferenz{\neqn{u}{v} \vee R}{\sigma(R)}{if
     $\sigma{} = \mw{mgu}(u,v)$ and $\sigma(\neqn{u}{v})$ is eligible
     for resolution.}
   
 \item \emph{Superposition into negative literals}\index{superposition}:
   
   \bigskip (SN) \GInferenz{\eqn{s}{t} \vee S \phantom{ae}
     \neqn{u}{v} \vee R}{\sigma(\neqn{u[p\leftarrow t]}{v} \vee S
     \vee R)}{if $\sigma=mgu(u|_p,s)$, $\sigma(s)\not<\sigma(t)$,
     $\sigma(u)\not<\sigma(v)$, $\sigma(\eqn{s}{t})$ is eligible for
     paramodulation, $\sigma(\neqn{u}{v})$ is eligible for resolution,
     and $u|_p \notin V$.}
   
 \item \emph{Superposition into positive literals}\index{superposition}:
   
   \bigskip (SP) \GInferenz{\eqn{s}{t} \vee S \phantom{ae}
     \eqn{u}{v} \vee R}{\sigma(\eqn{u[p\leftarrow t]}{v} \vee S \vee
     R)}{if $\sigma=mgu(u|_p,s)$, $\sigma(s)\not<\sigma(t)$,
     $\sigma(u)\not<\sigma(v)$, $\sigma(\eqn{s}{t})$ is eligible for
     paramodulation, $\sigma(\neqn{u}{v})$ is eligible for
     resolution, and $u|_p \notin V$.}
   
 \item \emph{Equality factoring}\index{equality factoring}:
   
   \bigskip (EF) \GInferenz{\eqn{s}{t} \vee \eqn{u}{v} \vee
     R}{\sigma(\neqn{t}{v} \vee \eqn{u}{v} \vee R)}{if
     $\sigma=mgu(s,u)$, $\sigma(s) \not> \sigma(t)$ and
     $\sigma(\eqn{s}{t})$ eligible for paramodulation.}
   
   
 \item \emph{Rewriting of negative literals}\index{rewriting}:
   
   \bigskip (RN) \CInferenz{\eqn{s}{t} \phantom{ae}
     \neqn{u}{v} \vee R} {\eqn{s}{t}
     \phantom{ae} \neqn{u[p\leftarrow{} \sigma(t)]}{v} \vee R} {if
     $u|_p = \sigma(s)$ and $\sigma(s)>\sigma(t)$.}
   
   
 \item \emph{Rewriting of positive
     literals}\index{rewriting}\footnote{A stronger version of (RP)
     is proven to maintain completeness for Unit and Horn problems
     and is generally believed to maintain completeness for the
     general case as well~\cite{Bachmair:personal-98}.  However, the
     proof of completeness for the general case seems to be rather
     involved, as it requires a very different clause ordering than
     the one introduced~\cite{BG94}, and we are not aware of any
     existing proof in the literature. The variant rule allows
     rewriting of maximal terms of maximal literals under certain
     circumstances:
     
     \medskip
     (RP') \CInferenzFoot{\eqn{s}{t} \phantom{ae}
       \eqn{u}{v} \vee R} {\eqn{s}{t}
       \phantom{ae} \eqn{u[p\leftarrow{} \sigma(t)]}{v} \vee R} {if
       $u|_p = \sigma(s)$, $\sigma(s)>\sigma(t)$ and
       if \eqn{u}{v} is not eligible for
       resolution or $u \not> v$ or $p \not=
       \lambda$ or $\sigma$ is not a variable renaming.}
     
     \medskip \noindent This stronger rule is implemented successfully
     by both E\index{E (theorem prover)} and
     SPASS\index{SPASS}~\cite{Weidenbach:personal-99}.}:
   
   \bigskip (RP) \CInferenz{\eqn{s}{t} \phantom{ae} \eqn{u}{v} \vee R}
   {\eqn{s}{t} \phantom{ae} \eqn{u[p\leftarrow{} \sigma(t)]}{v} \vee
     R} {if $u|_p = \sigma(s)$, $\sigma(s)>\sigma(t)$, and if
     \eqn{u}{v} is not eligible for resolution or $u \not> v$ or $p
     \not= \lambda$.}
   
 \item \emph{Clause subsumption}\index{subsumption}:
   
   \bigskip (CS) \CInferenz{C \phantom{ae} \sigma(C \vee R)}{C}{where
     $C$ and $R$ are arbitrary (partial) clauses and $\sigma$ is a
     substitution.} 
   
 \item \emph{Equality subsumption}\index{subsumption}:
   
   \bigskip (ES) \CInferenz{\eqn{s}{t} \phantom{ae}
     \eqn{u[p\leftarrow{} \sigma(s)]}{u[p\leftarrow{}
       \sigma(t)]\vee R}}{\eqn{s}{t}}{}
   
 \item \emph{Positive simplify-reflect\index{simplify-reflect}\footnote{In
       practice, this rule is only applied if $\sigma(s)$ and
       $\sigma(t)$ are $>$-incomparable -- in all other cases this
       rule is subsumed by (RN) and the deletion of resolved literals
       (DR).}}:
   
   \bigskip (PS) \CInferenz{\eqn{s}{t} \phantom{ae}
     \neqn{u[p\leftarrow{} \sigma(s)]}{u[p\leftarrow{}
       \sigma(t)]\vee R}}{\eqn{s}{t} \phantom{aee} R}{}
       
   
 \item \emph{Negative simplify-reflect\index{simplify-reflect}}
   
   \bigskip (NS) \CInferenz{\neqn{s}{t} \phantom{ae}
     \eqn{\sigma(s)}{\sigma(t)\vee R}}{\eqn{s}{t} \phantom{aee} R}{}
   
 \item \emph{Contextual (top level) simplify-reflect\index{contextual
       simplify-reflect}}
   
   \bigskip (CSR) \CInferenz{\sigma(C \vee R \vee \doteqn{s}{t})
     \hspace{2em} C \vee \overline{\doteqn{s}{t}}}{\sigma(C \vee R)
     \hspace{5em} C \vee \overline{\doteqn{s}{t}}}
   {where $\overline{\doteqn{s}{t}}$ is the negation of $\doteqn{s}{t}$
     and $\sigma$ is a substitution} 
     
  
 \item \emph{Tautology deletion}\index{tautology deletion}:
   
   \bigskip (TD) \CInferenz{C}{}{if C is a tautology\footnotemark{}.}
   \footnotetext{This rule can only be implemented approximately, as
     the problem of recognizing tautologies is only semi-decidable in
     equational logic. Current versions of 
     E try to detect tautologies by checking if the ground-completed
     negative literals imply at least one of the positive literals, as
     suggested in~\cite{NN:RTA-93}.}
   
 \item \emph{Deletion of duplicate literals}:
   
   \bigskip (DD) \CInferenz{\eqn{s}{t} \vee \eqn{s}{t} \vee
     R}{\eqn{s}{t} \vee R}{}
   
 \item \emph{Deletion of resolved literals}:
   
   \bigskip (DR) \CInferenz{\neqn{s}{s} \vee R}{R}{}
   
 \item \emph{Destructive equality resolution}\index{equality
     resolution}:
   
   \bigskip (DE) \CInferenz{\neqn{x}{y} \vee R}{\sigma(R)}{if $x,y \in
     V, \sigma = mgu(x,y)$}
   

\item \emph{Introduce definition\index{clause splitting}}

  \bigskip (ID) \CInferenz{R \vee S}{d \vee R \phantom{ae} \neg d \vee
    S}{if $R$ and $S$ do not share any variables, $d \in D$ has not
    been used in a previous definition, $R$ does not contain any
    symbol from $D$, and $S$ does not consist of only symbols from
    $D$}

\item \emph{Apply definition\index{clause splitting}}

  \bigskip (AD) \CInferenz{\sigma (d \vee R) \phantom{ae} R \vee S}
  {\sigma(d \vee R) \phantom{ae} \neg d \vee S}{if $\sigma$ is a
    variable renaming, $R$ and $S$ do not share
    any variables, $d \in D$, $R$ does not contain any symbol from
    $D$, and $S$ does not consist of only symbols from $D$}

   
 \end{itemize}
 We write $\mathbf{SP}(N)$ to denote the set of all clauses that can
 be generated with one generating inference from $I$ on a set of
 clauses $N$, $\mathcal{D}_{SP}$ to denote the set of all
 \textbf{SP}-derivations, and $\mathcal{D}_{\overline{SP}}$ to denote
  the set of all finite \textbf{SP}-derivations.
  
  \hfill$\blacktriangleleft$
\end{definition}



As \textbf{SP} only removes clauses that are \emph{composite} with
respect to the remaining set of clauses, the calculus is complete. For
the case of unit clauses, it degenerates into \emph{unfailing}
completion~\cite{BDP89} as implemented in DISCOUNT\index{DISCOUNT}. E
can also simulate the positive unit strategy for Horn clauses
described in~\cite{Dershowitz:IJCAI-91} using appropriate selection
functions.

Contrary to e.g.\ SPASS\index{SPASS}, E does not implement special
rules for non-equa\-tio\-nal literals or sort theories. Non-equational
literals are encoded as equations and dealt with accordingly.



\subsection{Proof Procedure}
\label{sec:procedure}

\begin{figure}[hp]
  \begin{center}
    \tt
    \begin{tabbing}
      \quad\quad \= \quad\quad \= \quad\quad \=  \quad\quad \= \quad\quad \= \kill
      \# Input: Axioms in U, P is empty\\
      while U $\not= \emptyset{}$ begin\+\\
        c := select(U)\\
        U := U $\backslash$ \{c\}\\

        \# Apply (RN), (RP), (NS), (PS), (CSR), (DR), (DD), (DE)\\
        simplify(c,P)\\
        \# Apply (CS), (ES), (TD)\\
        if c is trivial or subsumed by P then\+\\
          \# Delete/ignore c\-\\
        else if c is the empty clause then\+\\
          \# Success: Proof found\\
          stop\-\\
        else\+\\
          T := $\emptyset{}$ \# Temporary clause set\\
          foreach p $\in$ P do\+\\
            if c simplifies p \+\\
              P := P $\backslash$ \{p\}\\
              U := U $\backslash$ \{d|d \mbox{ is direct descendant of } p\}\\ 

              T := T $\cup$ \{p\}\-\\
            done\-\\
          end\\
          P := P $\cup$ \{c\}\\
          T := T $\cup$ e-resolvents(c)    \# (ER)\\
          T := T $\cup$ e-factors(c)       \# (EF)\\
          T := T $\cup$ paramodulants(c,P) \# (SN), (SP)\\
          foreach p $\in$ T do\+\\
            \# Apply efficiently implemented subset of (RN),\\
            \# (RP), (NS), (PS), (CSR), (DR), (DD), (DE)\\
            p := cheap\_simplify(p, P)\\
            \# Apply (TD) or efficient approximation of it\\
            if p is trivial\+\\
              \# Delete/ignore p\-\\
            else\+\\
               U := U $\cup$ cheap\_simplify(p, P)\-\\
            fi\-\\
          end\-\\
        fi\-\\        
      end\\
      \# Failure: Initial U is satisfiable, P describes model
    \end{tabbing}
    \normalfont
    \caption{Main proof procedure of E}
    \label{fig:procedure}\index{proof procedure}
  \end{center}
\end{figure}

Fig.~\ref{fig:procedure} shows a (slightly simplified) pseudocode
sketch of the quite straightforward proof procedure of E. The set of
all clauses is split into two sets, a set \texttt{P} of
\emph{processed} clauses and a set \texttt{U} of \emph{unprocessed}
clauses. Initially, all input clauses are in in \texttt{U}, and
\texttt{P} is empty. The algorithm selects a new clause from
\texttt{U}, simplifies it w.r.t. to \texttt{P}, then uses it to
back-simplify the clauses in \texttt{P} in turn. It then performs
equality factoring, equality resolution and superposition between the
selected clause and the set of processed clauses. The generated
clauses are added to the set of unprocessed clauses. The process stops
when the empty clause is derived or no further inferences are
possible.

The proof search is controlled by three major parameters: The term
ordering (described in section~\ref{sec:options:orderings}), the
literal selection function, and the order in which the \texttt{select}
operation selects the next clause to process. 

E implements two different classes of term orderings, lexicographic
term orderings and Knuth-Bendix orderings. A given ordering is
determined by instantiating one of the classes with a variety of
parameters (described in section~\ref{sec:options:orderings}).

Literal selection currently is done according to one of more than 50
predefined functions. Section~\ref{sec:options:strategies} describes
this feature.

Clause selection is determined by a heuristic evaluation function,
which conceptually sets up a set of priority queues and a weighted
round robin scheme that determines from which queue the next clause is
to be picked. The order within each queue is determined by a priority
function (which partitions the set of unprocessed clauses into one or
more subsets) and a heuristic evaluation function, which assigns a
numerical rating to each clause.  Section~\ref{sec:options:heuristics}
describes the user interface to this mechanism.



\section{Usage}
\label{sec:options}


\subsection{Search Control Heuristics}
\label{sec:options:heuristics}

Search control heuristics define the order in which the prover
considers newly generated clauses. A heuristic is defined by a set of
\emph{clause evaluation functions} and a selection scheme which defines
how many clauses are selected according to each evaluation function. A
clause evaluation function consists of a \emph{priority function} and
an instance of a generic \emph{weight function}.

\subsubsection{Priority functions}

Priority functions define a partition on the set of clauses.  A single
clause evaluation consists of a priority (which is the first selection
criteria) and an evaluation. Priorities are usually \emph{not} suitable
to encode heuristical control knowledge, but rather are used to
express certain elements of a search strategy, or to restrict the
effect of heuristic evaluation functions to certain classes of
clauses. It is quite trivial to add a new priority function to E, so
at any time there probably exist a few not yet documented here.

Syntactically, a large subset of currently available priority
functions is described by the following rule:

\begin{verbatim}
<prio-fun> ::= PreferGroundGoals ||
               PreferUnitGroundGoals ||
               PreferGround ||
               PreferNonGround ||
               PreferProcessed ||
               PreferNew ||
               PreferGoals ||
               PreferNonGoals ||
               PreferUnits ||
               PreferNonUnits ||
               PreferHorn ||
               PreferNonHorn ||
               ConstPrio ||
               ByLiteralNumber ||
               ByDerivationDepth ||
               ByDerivationSize ||
               ByNegLitDist ||
               ByGoalDifficulty ||
               SimulateSOS||
               PreferHorn||
               PreferNonHorn||
               PreferUnitAndNonEq||
               DeferNonUnitMaxEq||
               ByCreationDate||
               PreferWatchlist||
               DeferWatchlist
\end{verbatim}

The priority functions are interpreted as follows:

\begin{description}
\item[\texttt{PreferGroundGoals}:] Always prefer ground goals (all
  negative clauses without variables), do not differentiate between
  all other clauses.
\item[\texttt{PreferUnitGroundGoals}:] Prefer unit ground goals.
\item[\texttt{PreferGround}:] Prefer clauses without variables.
\item[\texttt{PreferNonGround}:] Prefer clauses with variables.
\item[\texttt{PreferProcessed}:] Prefer clauses that have already been
  processed once and have been eliminated from the set of processed
  clauses due to interreduction (forward contraction).
\item[\texttt{PreferNew}:] Prefer new clauses, i.e. clauses that are
  processed for the first time.
\item[\texttt{PreferGoals}:] Prefer goals (all negative clauses).
\item[\texttt{PreferNonGoals}:] Prefer non goals, i.e. facts with at
  least one positive literal.
\item[\texttt{PreferUnits}:] Prefer unit clauses (clauses with one
  literal).
\item[\texttt{PreferNonUnits}:] Prefer non-unit clauses.
\item[\texttt{PreferHorn}:] Prefer Horn clauses (clauses with no more
  than one positive literals).
\item[\texttt{PreferNonHorn}:] Prefer non-Horn clauses. 
\item[\texttt{ConstPrio}:] Assign the same priority to all clauses.
\item[\texttt{ByLiteralNumber}:] Give a priority according to the
  number of literals, i.e. always prefer a clause with fewer literals
  to one with more literals.
\item[\texttt{ByDerivationDepth}:] Prefer clauses which have a short
  derivation depth, i.e. give a priority based on the length of the
  longest path from the clause to an axiom in the derivation
  tree. Counts generating inferences only.
\item[\texttt{ByDerivationSize}:] Prefer clauses which have been
  derived with a small number of (generating) inferences.
\item[\texttt{ByNegLitDist}:] Prefer goals to non-goals. Among goals,
  prefer goals with fewer literals and goals with ground literals
  (more exactly: the priority is increased by 1 for a ground literal
  and by 3 for a non-ground literal. Clauses with lower values are
  selected before clauses with higher values).
\item[\texttt{ByGoalDifficulty}:] Prefer goals to non-goals. Select
  goals based on a simple estimate of their difficulty: First unit
  ground goals, then unit goals, then ground goals, then other goals.
\item[\texttt{SimulateSOS}:] Use the priority system to simulate
  Set-Of-Support. This prefers all initial clauses and all
  Set-Of-Support clauses. Some non-SOS-clauses will be generated, but
  not selected for processing. This is neither well tested nor
  a particularly good fit with E's calculus, but can be used as one
  among many heuristics. If you try a pure SOS strategy, you also
  should set \texttt{--restrict-literal-comparisons} and run the
  prover without literal selection enabled.
\item[\texttt{PreferHorn}:] Prefer Horn clauses (note: includes units).
\item[\texttt{PreferNonHorn}:] Prefer non-Horn clauses.
\item[\texttt{PreferUnitAndNonEq}:] Prefer all unit clauses and all
  clauses without equational literal. This was an attempt to model
  some restricted calculi used e.g.~in Gandalf~\cite{Tammet:JAR-97},
  but did not quite work out.
\item[\texttt{DeferNonUnitMaxEq}:] Prefer everything except for
  non-unit clauses with a maximal equational literal (``Don't
  paramodulate if its to expensive''). See above, same result.
\item[\texttt{ByCreationDate}:] Return the creation date of the clause
  as priority. This imposes a FIFO equivalence class on
  clauses. Clauses generated from the same given clause are grouped
  together (and can be ordered with any evaluation function among each
  other).
\item[\texttt{PreferWatchlist}] Prefer clauses on the watchlist
  (see~\ref{sec:options:watchlist}).
\item[\texttt{DeferWatchlist}] Defer clauses on the watchlist (see
  above).
\end{description}

Please note that careless use of certain priority functions can make
the prover incomplete for the general case.


\subsubsection{Generic Weight Functions}

Generic weight functions are templates for functions taking a clause
and returning a weight (i.e. an estimate of the usefulness) for it,
where a lower weight means that the corresponding clause should be
processed before a clause with a higher weight.  A generic weight
function is combined with a priority function and instantiated with a
set of parameters to yield a clause evaluation function.

You can specify an instantiated generic weight function as described
in this rule\footnote{Note that there now are many additional generic
  weight functions not yet documented.}:

\small
\begin{verbatim}
<weight-fun> ::= Clauseweight '(' <prio-fun> ', <int>, <int>,
                                  <float> ')'                    ||
                 Refinedweight '(' <prio-fun> ', <int>, <int>,
                                   <float>, <float>, <float> ')' ||
                 Orientweight '(' <prio-fun>, <int>, <int>,       
                                  <float>, <float>, <float> ')'  ||
                 Simweight '(' <prio-fun>, <float>, <float>,        
                               <float>, <float> ')'              ||
                 FIFOWeight '(' <prio-fun> ')'                   ||
                 LIFOWeight '(' <prio-fun> ')'                   ||
                 FunWeight '(' <prio-fun> ', <int>, <int>,
                               <float>, <float>, <float> 
                               (, <fun> : <posint> )* ')'        ||
                 SymOffsetWeight '(' <prio-fun> ', <int>, <int>,
                                     <float>, <float>, <float> 
                                    (, <fun> : <int> )* ')'
\end{verbatim}
\normalsize

\medskip
\noindent{}\texttt{Clauseweight(prio, fweight, vweight, pos\_mult)}:
This is the basic symbol counting heuristic. Variables are counted
with weight \texttt{vweight}, function symbols with weight
\texttt{fweight}. The weight of positive literals is multiplied by
\texttt{pos\_mult} before being added into the final weight.

\medskip
\begin{sloppypar}
\noindent{}\texttt{Refinedweight(prio, fweight, vweight,
    term\_pen, lit\_pen, pos\_mult)}: 
This weight function is very similar to the first one. It differs only
in that it takes the effect of the term ordering into account. In
particular, the weight of a term that is maximal in its literal is
multiplied by \texttt{term\_pen}, and the weight of maximal
literals is multiplied by \texttt{lit\_pen}.
\end{sloppypar}

\medskip
\begin{sloppypar}
\noindent{}\texttt{Orientweight(prio, fweight, vweight,
    term\_pen, lit\_pen, pos\_mult)}: This weight function is a
slight variation of \texttt{Refinedweight()}. In this case, the weight
of \emph{both} terms of an unorientable literal is multiplied by a
penalty \texttt{term\_pen}.
\end{sloppypar}

\medskip
\begin{sloppypar}
\noindent{}\texttt{Simweight(prio, equal\_weight,
  vv\_clash, vt\_clash, tt\_clash)}: This weight function is intended
to return a low weight for literals in which the two terms are very
similar. It does not currently work very well even for unit clauses --
RTFS (in \texttt{<che\_simweight.c>}) to find out more.
\end{sloppypar}

\medskip
\begin{sloppypar}
\noindent{}\texttt{FIFOWeight(prio)}:
This weight function assigns weights that increase in a strictly
monotonic manner, i.e. it realizes a \emph{first-in/first-out}
strategy if used all by itself. This is the most obviously fair
strategy.
\end{sloppypar}

\medskip
\begin{sloppypar}
\noindent{}\texttt{LIFOWeight(prio)}:
This weight function assigns weights that decrease in a strictly
monotonic manner, i.e. it realizes a \emph{last-in/first-out}
strategy if used all by itself (which, of course, would be unfair and
result in an extremely incomplete prover).
\end{sloppypar}

\medskip
\begin{sloppypar}
  \noindent{}\texttt{FunWeight(prio, fweight, vweight,
    term\_pen, lit\_pen, pos\_mult, fun:fweight \ldots)}: 
  This evaluation function is a variant of \texttt{Refinedweight}. The
  first 6 parameter are identical in meaning. The function takes an
  arbitrary number of extra parameters of the form
  \texttt{fun:fweight}, where \texttt{fun} is any valid function
  symbol, and \texttt{fweight} is a non-negative integer. The extra
  weight assignments will overwrite the default weight for the listed
  function symbol.
\end{sloppypar}

\medskip
\begin{sloppypar}
  \noindent{}\texttt{SymOffsetWeight(prio, fweight, vweight,
    term\_pen, lit\_pen, pos\_mult, fun:fweight \ldots)}: 
  This evaluation function is similar to \texttt{FunWeight}. The first
  6 parameter are identical in meaning.  The extra arguments allow
  both positive and negative values, and are used as once-off weight
  modifiers added to the weight of all clauses that contain the
  defined symbol.
\end{sloppypar}


\subsubsection{Clause Evaluation Functions}
\index{clause evaluation}

A clause evaluation function is constructed by instantiating a generic
weight function. It can either be specified directly, or specified and
given a name for later reference at once:

 \begin{verbatim}
<eval-fun>          ::= <ident>         ||
                        <weight-fun>    ||
                        <eval-fun-def>  
<eval-fun-def>      ::= <ident> = <weight-fun>
<eval-fun-def-list> ::= <eval-fun-def>*
\end{verbatim}

Of course a single identifier is only a valid evaluation function if
it has been previously defined in a \texttt{<eval-fun-def>}. It is
possible to define the value of an identifier more than once, in which
case later definitions take precedence to former ones.

Clause evaluation functions can be be defined on the command line with
the \texttt{-D} (\texttt--{define-weight-function}) option, followed
by a \texttt{<eval-fun-def-list>}.

\begin{example}
\begin{verbatim}
    eprover -D"ex1=Clauseweight(ConstPrio,2,1,1) \
               ex2=FIFOWeight(PreferGoals)" ...
\end{verbatim}
  sets up the prover to know about two evaluation function
  \texttt{ex1} and \texttt{ex2} (which supposedly will be used later on
  the command line to define one or more heuristics). The double
  quotes are necessary because the brackets and the commas are
  special characters for most shells
\end{example}

There are a variety of clause evaluation functions predefined in
the variable \texttt{DefaultWeightFunctions}, which can be found in
\texttt{che\_proofcontrol.c}. See also
sections~\ref{sec:options:watchlist} and ~\ref{sec:options:learning},
which cover some of the more complex weight functions of E.


\subsubsection{Heuristics}

A heuristic defines how many selections are to be made according to
one of several clause evaluation functions. Syntactically,

\begin{verbatim}
<heu-element>   ::= <int> '*' <eval-fun>
<heuristic>     ::= '(' <heu-element> (,<heu-element>)* ')' ||
                    <ident>
<heuristic-def> ::= <ident> = <heuristic> ||
                    <heuristic>
\end{verbatim}

As above, a single identifier is only a valid heuristic if it has been
defined in \texttt{<heuristic-def>} previously. A
\texttt{<heuristic-def>} which degenerates to a simple heuristic
defines a heuristic with name \texttt{Default} (which the prover will
automatically choose if no other heuristic is selected with the
\texttt{-x} (\texttt{--expert-heuristic}).

\begin{example}
  To continue the above example,
\begin{verbatim}
    eprover -D"ex1=Clauseweight(ConstPrio,2,1,1) \
               ex2=FIFOWeight(PreferGoals)" 
            -H"new=(3*ex1,1*ex2)" \
            -x new LUSK3.lop
\end{verbatim}
  will run the prover on a problem file named LUSK3.lop with a
  heuristic that chooses 3 out of every 4 clauses according to a
  simple symbol counting heuristic and the last clause first among
  goals and then among other clauses, selecting by order of creation
  in each of these two classes.
\end{example}


\subsection{Term Orderings}
\label{sec:options:orderings}
\index{term ordering}

E currently supports two families of orderings: The
\emph{Knuth-Bendix-Ordering} (KBO), which is used by default, and the
\emph{Lexicographical Path Ordering} (LPO). The KBO is weight-based
and uses a precedence on function symbols to break ties. Consequently,
to specify a concrete KBO, we need a weight function that assigns a
weight to all function symbols, and a precedence on those symbols.

The LPO is based on a lexicographic comparison of symbols and
subterms, and is fully specified by giving just a precedence.

Currently it is possible to explicitly specify an arbitrary
(including incomplete or empty) precedence, or to use one of several
precedence generating schemes. Similarly, there is a number of
predefined weight function and the ability to assign arbitrary weights
to function and predicate symbols.

The simplest way to get a reasonable term ordering is to specify
\emph{automatic} ordering selection using the \texttt{-tAuto} option. 

\noindent
Options controlling the choice of term ordering:\\[1ex]
\begin{supertabular}{lp{9.5cm}}
  \multicolumn{2}{l}{\texttt{-term-ordering=<arg>}}\\
  \texttt{-t<arg>} & Select a term ordering class (or automatic
  selection). Supported arguments are at least \texttt{LPO},
  \texttt{LPO4} (for a much faster new implementation of LPO),
  \texttt{KBO}, and \texttt{Auto}. If \texttt{Auto} is selected, all
  aspects of the term ordering are fixed, additional
  options will be (or at least should be) silently ignored.\\[1ex]

  \multicolumn{2}{l}{\texttt{\texttt{--order-precedence-generation=<arg>}}}\\
  \texttt{-G <arg>} & Select a precedence generation scheme (see
  below).\\[1ex]

  \multicolumn{2}{l}{\texttt{\texttt{--order-weight-generation=<arg>}}}\\
  \texttt{-w <arg>} & Select a symbol weight function (see
  below).\\[1ex]

  \multicolumn{2}{l}{\texttt{\texttt{--order-constant-weight=<arg>}}}\\
  \texttt{\texttt{-c <arg>}} & Modify any symbol weight function by
  assigning a special weight to constant function symbols.\\[1ex]

  \multicolumn{2}{l}{\texttt{--precedence[=<arg>]}}\\
  & Describe a (partial) precedence for the term ordering. The argument
   is a comma-separated list of precedence chains, where a precedence
   chain is a list of function symbols (which all have to appear in
   the proof problem), connected by \texttt{>}, \texttt{<}, or
  \texttt{=} (to denote equivalent symbols).
  If this option is used in connection with
  \texttt{--order-precedence-generation}, the partial ordering will be
  completed using the selected method, otherwise the prover runs with
  a non-ground-total ordering. The option without the optional
  argument is equivalent to \texttt{--precedence=} (the empty
  precedence). There is a drawback to using \texttt{--precedence}:
  Normally, total precedences are represented by mapping symbols to a
  totally ordered set (small integers) which can be compared using
  standard machine instructions. The used data structure is linear in
  the number $n$ of function symbols. However, if \texttt{--precedence}
  is used, the prover allocates (and completes) a $n\times n$ lookup
  table to efficiently represent an arbitrary partial ordering. If $n$
  is very big, this matrix takes up significant  space, and takes a
  long time to compute in the first place. This is unlikely to be a
  problem unless there are at least hundreds of symbols.\\[1ex]

  \multicolumn{2}{l}{\texttt{--order-weights=<arg>}}\\
  & Give explicit weights to function symbols. The argument syntax is
  a comma-separated list of items of the form \texttt{f:w}, where
  \texttt{f} is a symbol from the specification, and \texttt{w} is a
  non-negative integer. Note that at best very simple checks are
  performed, so you can specify weights that do not obey the KBO
  weight constraints. Behaviour in this case is undefined. If all your
  weights are positive, this is unlikely to happen.

  Since KBO needs a total weight function, E always uses a weight
  generation scheme in addition to the user-defined options. You may
  want to use \texttt{-wconstant} for predictable behaviour.\\[1ex]

  \multicolumn{2}{l}{\texttt{--lpo-recursion-limit[=<arg>]}}\\
  & Limits the recursion depth of LPO comparison. This is useful in rare
  cases where very large term comparisons can lead to stack overflow
  issues. It does not change completeness, but may lead to unnecessary
  inferences in rare cases (Note: By default, recursion depth is
  limited to 1000. To get effectively unlimited recursion depth, use
  this option with an outrageously large argument. Don't forget to
  increase process stack size with \texttt{limit/ulimit} from your
  favourite shell).\\
\end{supertabular}

\subsubsection{Precedence Generation Schemes}

Precedence generation schemes are based on syntactic features of the
symbol and the input clause set, like symbol arity or number of
occurrences in the formula. At least the following options are
supported as argument to \texttt{--order-precedence-generation}:

\begin{description}
\item[\texttt{unary\_first}:] Sort symbols by arity, with the
  exception that unary symbols come first. Frequency is used as a
  tie breaker (rarer symbols are greater).
\item[\texttt{unary\_freq}:] Sort symbols by frequency (rarer symbols
  are bigger), with the exception that unary symbols come first. Yes,
  this should better be named \texttt{unary\_invfreq} for
  consistency, but is not\ldots
\item[\texttt{arity}:] Sort symbols by arity (symbols with higher
  arity are larger).
\item[\texttt{invarity}:] Sort symbols by arity (symbols with higher
  arity are smaller).
\item[\texttt{const\_max}:] Sort symbols by arity (symbols with higher
  arity are larger), but make constants the largest symbols. This is allegedly
  used by SPASS~\cite{Weidenbach:SPASS-2001} in some configurations.
\item[\texttt{const\_min}:] Sort symbols by arity (symbols with higher
  arity are smaller), but make constants the smallest symbols.
  Provided for reasons of symmetry.
\item[\texttt{freq}:] Sort symbols by frequency (frequently occurring
  symbols are larger). Arity is used as a tie breaker.
\item[\texttt{invfreq}:] Sort symbols by frequency (frequently occurring
  symbols are smaller). In our experience, this is one of the best
  general-purpose precedence generation schemes.
\item[\texttt{invfreqconstmin}:] Same as \texttt{invfreq}, but make
  constants always smaller than everything else.
\item[\texttt{invfreqhack}:] As \texttt{invfreqconstmin}, but unary
  symbols with maximal frequency become largest.
\end{description}


\subsubsection{Weight Generation Schemes}

Weight generation schemes are based on syntactic features of the
symbol and the input clause set, or on the predefined
\emph{precedence}. The following options are available for
\texttt{--order-weight-generation}.


\begin{description}
\item[\texttt{firstmaximal0}:] Give the same arbitrary (positive)
  weight to all function symbols except to the first maximal one
  encountered (order is arbitrary), which is given weight 0.
\item[\texttt{arity}:] Weight of a function symbol $f|_n$ is $n+1$,
  i.e. its arity plus one.
\item[\texttt{aritymax0}:] As \texttt{arity}, except that the
  first maximal symbol is given weight 0.
\item[\texttt{modarity}:] Weight of a function symbol $f|_n$ is $n+c$,
  where $c$ is a positive constant (\texttt{W\_TO\_BASEWEIGHT}, which
  has been 4 since the dawn of time).
\item[\texttt{modaritymax0}:] As \texttt{modarity}, except that the
  first maximal symbol is given weight 0.
\item[\texttt{aritysquared}:] Weight of a symbol $f|_n$ is $n^2+1$.
\item[\texttt{aritysquaredmax0}:] As \texttt{aritysquared}, except
  that the first maximal symbol is given weight 0.
\item[\texttt{invarity}:] Let $m$ be the largest arity of any symbol
  in the signature.  Weight of a symbol $f|_n$ is $m-n+1$.
\item[\texttt{invaritymax0}:] As \texttt{invarity}, except
  that the first maximal symbol is given weight 0.
\item[\texttt{invaritysquared}:] Let $m$ be the largest arity of any symbol
  in the signature.  Weight of a symbol $f|_n$ is $m^2-n^2+1$.
\item[\texttt{invaritysquaredmax0}:] As \texttt{invaritysquared},
  except that the first maximal symbol is given weight 0.
\item[\texttt{precedence}:] Let $<$ be the (pre-determined) precedence
  on function symbols $F$ in the problem. Then the weight of $f$ is
  given by $|{g|g<f}|+1$ (the number of symbols smaller than $f$ in
  the precedence increased by one).
\item[\texttt{invprecedence}:] Let $<$ be the (pre-determined)
  precedence on function symbols $F$ in the problem. Then the weight
  of $f$ is given by $|{g|f<g}|+1$ (the number of symbols larger than
  $f$ in the precedence increased by one).
\item[\texttt{freqcount}:] Make the weight of a symbol the number of
  occurrences of that symbol in the (potentially preprocessed) input
  problem. 
\item[\texttt{invfreqcount}:]  Let $m$ be the number of occurrences of
  the most frequent symbol in the input problem. The weight of $f$ is
  $m$ minus he number of occurrences of $f$ in the input problem.
\item[\texttt{freqrank}:] Sort all function symbols by frequency of
  occurrence (which induces a total quasi-ordering). The weight of a
  symbol is the rank of it's equivalence class, with less frequent
  symbols getting lower weights.
\item[\texttt{invfreqrank}:] Sort all function symbols by frequency of
  occurrence (which induces a total quasi-ordering). The weight of a
  symbol is the rank of its equivalence class, with less frequent
  symbols getting higher weights.
\item[\texttt{freqranksquare}:] As \texttt{freqrank}, but weight is
  the square of the rank.
\item[\texttt{invfreqranksquare}:] As \texttt{invfreqrank}, but weight is
  the square of the rank.
\item[\texttt{invmodfreqrank}:] Sort all function symbols by frequency of
  occurrence (which induces a total quasi-ordering). The weight of an
   equivalence class is the sum of the cardinality of all smaller
  classes (+1). The weight of a symbol is the weight of its
  equivalence classes. Less frequent symbols get higher weights.
\item[\texttt{invmodfreqrankmax0}:] As \texttt{invmodfreqrank}, except
  that the first maximal symbol is given weight 0.
\item[\texttt{constant}:] Give the same arbitrary positive weight to
  all function symbols.
\end{description}


\subsection{Literal Selection Strategies}
\label{sec:options:strategies}

The superposition calculus allows the \emph{selection} of arbitrary
negative literals in a clause and only requires generating inferences
to be performed on these literals. E supports this feature and
implements it via manipulations of the literal ordering. Additionally,
E implements strategies that allow inferences into maximal positive
literals and selected negative literals. A selection strategy is
selected with the option \texttt{--literal-selection-strategy}.
Currently, at least the following strategies are implemented:

\begin{description}
\item[\texttt{NoSelection}:]
 Perform ordinary superposition without selection.
\item[\texttt{NoGeneration}:] Do not perform any generating
  inferences. This strategy is not complete, but applying it to a
  formula generates a normal form that does not contain any
  tautologies or redundant clauses.
\item[\texttt{SelectNegativeLiterals}:] Select all negative literals. For Horn
  clauses, this implements the maximal literal positive unit
  strategy~\cite{Dershowitz:IJCAI-91} previously realized separately in
  E.
\item[\texttt{SelectPureVarNegLiterals}:] Select the first negative literal of
  the form \eqn{X}{Y}.
\item[\texttt{SelectLargestNegLit}:] Select the largest negative literal (by
  symbol counting, function symbols count as 2, variables as 1).
\item[\texttt{SelectSmallestNegLit}:] As above, but select the smallest
  literal.
\item[\texttt{SelectDiffNegLit}:] Select the negative literal in which both
  terms have the largest size difference.
\item[\texttt{SelectGroundNegLit}:] Select the first negative ground literal
  for which the size difference between both terms is maximal.
\item[\texttt{SelectOptimalLit}:] If there is a ground negative literal, select
  as in the case of \texttt{SelectGroundNegLit}, otherwise as in
  \texttt{SelectDiffNegLit}.
\end{description}

Each of the strategies that do actually select negative literals has a
corresponding counterpart starting with \texttt{P} that additionally
allows paramodulation into maximal positive literals\footnote{Except
  for \texttt{SelectOptimalLit}, where the resulting strategy,
  \texttt{PSelectOptimalLit} will allow paramodulation into positive
  literals only if no ground literal has been selected.}.


\begin{example}
  Some problems become a lot simpler with the correct
  strategy. Try e.g.
\begin{verbatim}
eprover --literal-selection-strategy=NoSelection \
          GRP001-1+rm_eq_rstfp.lop
eprover --literal-selection-strategy=SelectLargestNegLit \
          GRP001-1+rm_eq_rstfp.lop
\end{verbatim}
  You will find the file \texttt{GRP001-1+rm\_eq\_rstfp.lop} in the
  \texttt{E/PROVER} directory.
\end{example}

As we aim at replacing the vast number of individual literal selection
functions with a more abstract mechanism, we refrain from describing
all of the currently implemented functions in detail. If you need
information about the set of implemented functions, run
\texttt{eprover -W none}. The individual functions are implemented and
somewhat described in \texttt{E/HEURISTICS/che\_litselection.h}.


\subsection{The Watchlist Feature}
\label{sec:options:watchlist}

Since public release 0.81, E supports a \emph{watchlist}. A watchlist
is a user-defined set of clauses. Whenever the prover
encounters\footnote{Clauses are checked against the watchlist after
  normalization, both when they are inserted into \texttt{U} or if
  they are selected for processing.} a clause that subsumes one or
more clauses from the watchlist, those clauses are removed from it.
The saturation process terminates if the watchlist is empty (or, of
course, if a saturated state or the empty clause have been reached).

There are two uses for a watchlist: To guide the proof search (using a
heuristic that prefers clauses on the watchlist), or to find purely
constructive proofs for clauses on the watchlist. 

If you want to guide the proof search, place clauses you believe to be
important lemmata onto the watchlist. Also include the empty clause to
make sure that the prover will not terminate prematurely.  You can
then use a clause selection heuristic that will give special
consideration to clauses on the watchlist. This is currently supported
via the \emph{priority functions} \texttt{PreferWatchlist} and
\texttt{DeferWatchlist}. A clause evaluation function using
\texttt{PreferWatchlist} will always select clauses which subsume
watchlist clauses first. Similarly, using \texttt{DeferWatchlist} can
be used to put the processing of watchlist clauses off.

There is a predefined clause selection heuristic \texttt{UseWatchlist}
(select it with \texttt{-xUseWatchlist}) that will make sure that
watchlist clauses are selected relatively early. It is a strong
general purpose heuristic, and will maintain completeness of the
prover. This should allow easy access to the watchlist feature even
if you don't yet feel comfortable with specifying your own
heuristics. 

To generate constructive proofs of clauses, just place them on the
watch list and select output level 4 or greater (see
section~\ref{sec:output:lots}). Steps effecting the watch list will be
marked in the PCL2 output file. If you use the \emph{eproof} script
for proof output or run \emph{epclextract} of your own, subproof for
watchlist steps will be automatically extracted.

Note that this forward reasoning is not complete, i.e.\ the prover may
never generate a given watchlist clause, even if it would be trivial
to prove it via refutation.

Options controlling the use of the watch list:\\
\begin{tabular}{lp{5.8cm}}
  \texttt{--watchlist=<arg>} & Select a file containing the watch list
  clauses. Syntax should be the same syntax as your proof problem
  (E-LOP, TPTP or TSTP). Just write down a list of clauses.\\
  \texttt{--no-watchlist-simplification} & By default, watch list
  clauses are simplified with respect to the current set
  \texttt{P}. Use this option to disable the feature.
\end{tabular}


\subsection{Learning Clause Evaluation Functions}
\label{sec:options:learning}

E can use a knowledge base generated by analyzing many successful
proof attempts to guide its search, i.e.\ it can \emph{learn} what
kinds of clauses are likely to be useful for a proof and which ones
are likely superfluous. The details of the learning mechanism can be
found in~\cite{Schulz:Diss-2000,Schulz:KI-2001}. Essentially, an
inference protocol is analyzed, useful and useless clauses are
identified and generalized into \emph{clause patterns}, and the
resulting information is stored in a knowledge base. Later, new clauses
that match a pattern are evaluated accordingly.

\subsubsection{Creating Knowledge Bases}

An E knowledge base is a directory containing a number of files,
storing both the knowledge and configuration information. Knowledge
bases are generated with the tool \texttt{ekb\_create}. If no argument
is given, \texttt{ekb\_create} will create a knowledge base called
\texttt{E\_KNOWLEDGE} in the current directory.

You can run \texttt{ekb\_create -h} for more information about the
configuration. However, the defaults are usually quite sufficient.


\subsubsection{Populating Knowledge Bases}

The knowledge base contains information gained from clausal PCL2
protocols of E. In a first step, information from the protocol is
abstracted into a more compact form. A number of clauses is selected
as training examples, and annotations about there role are computed.
The result is a list of annotated clauses and a list of the axioms
(initial clauses) of the problem. This step can be performed using the
program \texttt{direct\_examples}\footnote{The name is an historical
  accident and has no significance anymore}. 

In a second step, the collected information is integrated into the
knowledge base. For this purpose, the program \texttt{ekb\_insert}
can be used. However, it is probably more convenient to use the single
program \texttt{ekb\_ginsert}, which directly extracts all pertinent
information from a PCL2 protocol and inserts it into a designated
knowledge base.

The program \texttt{ekb\_delete} will delete an example from a
knowledge base. This process is not particularly efficient, as the
whole knowledge base is first parsed.


\subsubsection{Using Learned Knowledge}

The knowledge in a knowledge base can be utilized by the two clause
evaluation functions \texttt{TSMWeight()} and \texttt{TSMRWeight()}.
Both compute a modification weight based on the learned knowledge, and
apply it to a conventional symbol-counting base weight (similar to
\texttt{Clauseweight()} for \texttt{TSMWeight()} and 
\texttt{Refinedweight()} for \texttt{TSMWeight()}. An example command
line is:

\texttt{eprover -x'(1*TSMWeight(ConstPrio, 1, 1, 2, flat,
  E\_KNOWLEDGE,}

\texttt{100000,1.0,1.0,Flat,IndexIdentity,100000,-20,20,-2,-1,0,2))'}

There are also two fully predefined learning clause selection
heuristics. Select them with \texttt{-xUseTSM1} (for some influence of
the learned knowledge) or \texttt{-xUseTSM2} (for a lot of influence
of the learned knowledge).



\subsection{Other Options}
\label{sec:options:others}


\section{Input Language}
\label{sec:language}

\subsection{LOP}

E natively uses E-LOP, a dialect of the LOP language designed for
SETHEO.At the moment, your best bet is to retrieve the LOP description
from the E web site~\cite{E:WWW-99} and/or check out the examples
available from it.  LOP is very close to Prolog, and E can usually
read many fully declarative Prolog files if they do not use arithmetic
or rely on predefined symbols. Plain SETHEO files usually also work
very well.  There are a couple of minor differences, however:

\begin{itemize}
\item \texttt{equal()} is an interpreted symbol for E. It normally
  does not carry any meaning for SETHEO (unless equality axioms are
  added).
\item SETHEO allows the same identifier to be used as a constant, a
  non-constant function symbol and a predicate symbol. E encodes all
  of these as ordinary function symbols, and hence will complain if a
  symbol is used inconsistently.
\item E allows the use of both \texttt{=} and \texttt{=>} as infix
  symbols for equality. \texttt{a=b} is equivalent to
  \texttt{equal(a,b)} for E.
\item E does not support constraints or SETHEO build-in symbols. This
  should not usually affect pure theorem proving tasks.
\item E normally treats procedural clauses exactly as it treats
  declarative clauses. Query clauses (clauses with an empty head and
  starting with \texttt{?-}, e.g. \texttt{?-$\sim$p(X), q(X).} can
  optionally be used to define the a set of \emph{goal clauses} (by
  default, all negative clauses are considered to be goals). At the
  moment, this information is only used for the initial set of support
  (with \texttt{--sos-uses-input-types}). Note that you can still
  specify arbitrary clauses as query clauses, since LOP supports
  negated literals.
\end{itemize}

\subsection{TPTP Format}

The TPTP~\cite{Sutcliffe:TPTP-WWW} is a library of problems for
automated theorem prover. Problems in the TPTP are written in TPTP
syntax. There are two major versions of the TPTP syntax, both of which
are supported by E.

Version 2\footnote{Version 1 allowed the specification of problems in
  clause normal form only. Version 2 is a conservative extension of
  version 1 and adds support for full first order formulas.} of the
TPTP syntax was used up for TPTP releases previous to TPTP~3.0.0.  The
current version 3 of the TPTP syntax, described in
\cite{GSCG:IJCAR-2006}, covers both input problems and both proof and
model output using one consistent formalism. It has been used as the
native format for TPTP releases since TPTP~3.0.0.

Parsing in TPTP format version 2 is enabled by the options
\texttt{--tptp-in}, \texttt{tptp2-in}, \texttt{--tptp-format} and
\texttt{--tptp2-format}. The last two options also select TPTP 2
format for the output of normal clauses during and after
saturation. Proof output will be in PCL2 format, however.

TPTP syntax version 3 is the currently recommended format. It is
supported by many provers, it is more consistent than the old TPTP
language, and it adds a number of useful features.  E supports TPTP-3
syntax with the options \texttt{--tstp-in} , \texttt{tptp3-in},
\texttt{--tstp-format} and \texttt{--tptp3-format}. The last two
options will also enable TPTP-3 format for proof output. Note that
many of E's support tools still require PCL2 format. Various tools for
processing TPTP-3 proof format are available via the TPTP web-site,
\url{http://www.tptp.org}.

In either TPTP format, clauses and formulas with TPTP type
\texttt{conjecture} or \texttt{negated-conjecture} (TPTP-3 only) are
considered goal clauses for the \texttt{--sos-uses-input-types}
option.



\section{Output\ldots or how to interpret what you see}
\label{sec:output}

E has several different output levels, controlled by the option
\texttt{-l} or \texttt{--output-level}. Level 0 prints nearly no
output except for the result. Level 1 is intended to give humans a
somewhat readable impression of what is going on inside the inference
engine.  Levels 3 to 6 output increasingly more information about the
inside processes in PCL2 format. At level 4 and above, a (large)
superset of the proof inferences is printed. You can use the
\texttt{epclextract} utility in \texttt{E/PROVER/} to extract a simple
proof object.

In Level 0 and 1, everything E prints is either a clause that is
implied by the original axioms, or a comment (or, very often, both).


\subsection{The Bare Essentials}
\label{sec:output:essentials}

In silent mode (\texttt{--output-level=0}, \texttt{-s} or
\texttt{--silent}), E will not print any output during saturation. It
will print a one-line comment documenting the state of the proof
search after termination. The following possibilities exist:

\begin{itemize}
\item The prover found a proof. This is denoted by the output string
\begin{verbatim}
# Proof found!
\end{verbatim}
\item The problem does not have a proof, i.e. the specification is
  satisfiable (and E can detect this):
\begin{verbatim}
# No proof found!
\end{verbatim}
  Ensuring the completeness of a prover is much harder than ensuring
  correctness. Moreover, proofs can easily be checked by analyzing the
  output of the prover, while such a check for the absence of proofs
  is rarely possible. I do believe that the current version of E is
  both correct and complete\footnote{Unless the prover runs out of
    memory (see below), the user selects an unfair strategy (in which
    case the prover may never terminate), or some strange and
    unexpected things happen.} but my belief in the former is stronger
  than my belief in the later\ldots...
\item A (hard) resource limit was hit. For memory this can be either
  due to a per process limit (set with \texttt{limit} or the prover
  option \texttt{--memory-limit}), or due to running out of virtual
  memory. For CPU time, this case is triggered if the per process CPU
  time limit is reached and signaled to the prover via a
  \texttt{SIGXCPU} signal. This limit can be set with \texttt{limit}
  or, more reliable, with the option \texttt{--cpu-limit}. The output
  string is one of the following two, depending on the exact reason
  for termination:
\begin{verbatim}
# Failure: Resource limit exceeded (memory)
# Failure: Resource limit exceeded (time)
\end{verbatim}
\item A user-defined limit was reached during saturation, and the
  saturation process was stopped gracefully. Limits include number of
  processed clauses, number of total clauses, and cpu time (as set
  with \texttt{--soft-cpu-limit}). The output string is
\begin{verbatim}
# Faiure: User resource limit exceeded!
\end{verbatim}
  \ldots and the user is expected to know which limit he selected.
\item Normally, E is complete. However, if the option
  \texttt{--delete-bad-limit} is given or if automatic mode in
  connection with a memory limit is used, E will periodically delete
  clauses it deems unlikely to be processed to avoid running out of
  memory. In this case, completeness cannot be ensured any more. This
  effect manifests itself extremely rarely. If it does, E will print
  the following string:
\begin{verbatim}
  # Failure: Out of unprocessed clauses!
\end{verbatim}
  This is roughly equivalent to Otter's \texttt{SOS empty} message.
\item Finally, it is possible to chose restricted calculi when
  starting E. This is useful if E is used as a normalization tool or
  as a preprocessor or lemma generator. In this case, E will print a
  corresponding message:
\begin{verbatim}
# Clause set closed under restricted calculus!
\end{verbatim}
\end{itemize}


\subsection{Impressing your Friends}
\label{sec:output:normal}

If you run E without selection an output level (or by setting it
explicitly to 1), E will print each non-tautological, non-subsumed
clause it processes as a comment. It will also print a hash
('\verb+#+') for each clause it tries to process but can prove to be
superfluous.

This mode gives some indication of progress, and as the output is
fairly restricted, does not slow the prover down too much.

For any output level greater than 0, E will also print statistical
information about the proof search and final clause sets. The data
should be fairly self-explaining.


\subsection{Detailed Reporting}
\label{sec:output:lots}

At output levels greater that 1, E prints certain inferences in PCL2
format\footnote{PCL2 is a proof output designed as a successor to
  PCL~\cite{DS94a,DS94b,DS96a}.} or TPTP-3 output format.  At level 2,
it only prints generating inferences. At level 4, it prints all
generating and modifying inferences, and at level 6 it also prints PCL
steps giving a lot of insight into the internal operation of the
inference engine.  This protocol is fairly readable and, from level 4
on can be used to check the proof with the utility \texttt{checkproof}
provided with the distribution.


\subsection{Requesting Specific Output}
\label{sec:output:results}

There are two additional kinds of information E can provide beyond the
normal output during proof search: Statistical information and final
clause sets (with additional information).

First, E can give you some technical information about the conditions
it runs under. 

The option \texttt{--print-pid} will make E printing its process id as
a comment, in the format \texttt{\# Pid: XXX}, where \texttt{XXX} is an
integer number. This is useful if you want to send signals to the
prover (in particular, if you want to terminate the prover) to control
it from the outside.

The option \texttt{-R} (\texttt{--resources-info}) will make E print a
summary of used system resources after graceful termination:

\begin{verbatim}
# User time                : 0.010 s
# System time              : 0.020 s
# Total time               : 0.030 s
# Maximum resident set size: 0 pages
\end{verbatim}

Most operating systems do not provide a valid value for the resident
set size and other memory-related resources, so you should probably
not depend on the last value to carry any meaningful information. The
time information is required by most standards and should be useful
for all tested operating systems.

E can be used not only as a prover, but as a normalizer for formulae
or as a lemma generator. In this cases, you will not only want to know
if E found a proof, but also need some or all of the derived clauses,
possibly with statistical information for filtering. This is supported
with the \texttt{--print-saturated} and \texttt{--print-sat-info}
options for E.

The option \texttt{--print-saturated} takes as its argument a string
of letters, each of which represents a part of the total set of
clauses E knows about. The following table contains the meaning of the
individual letters:

\begin{tabular}{lp{9cm}}
  \texttt{e} & Processed positive unit clauses (\emph{Equations}).\\ 
  \texttt{i} & Processed negative unit clauses (\emph{Inequations}).\\ 
  \texttt{g} & Processed non-unit clauses (except for the empty
  clause, which, if present, is printed separately). The above three
  sets are interreduced and all selected inferences between them have
  been computed.\\ 
  \texttt{E} & Unprocessed positive unit clauses.\\ 
  \texttt{I} & Unprocessed negative unit clauses.\\ 
  \texttt{G} & Unprocessed non-unit clause (this set may contain the
  empty clause in very rare cases).\\ 
  \texttt{a} & Print equality axioms (if equality is present in the
  problem). This letter prints axioms for reflexivity, symmetry, and
  transitivity, and a set of substitutivity axioms, one for each
  argument position of every function symbol and predicate symbol.\\
  \texttt{A} & As \texttt{a}, but print a single substitutivity axiom
  covering all positions for each symbol.\\
\end{tabular}

The short form, \texttt{-S}, is equivalent to
\texttt{--print-saturated=eigEIG}. If the option
\texttt{--print-sat-info} is set, then each of the clauses is followed
by a comment of the form \texttt{\# info(id, pd, pl, sc, cd, nl,
  no, nv)}. The following table explains the meaning of these values:

\begin{tabular}{lp{10cm}}
\texttt{id} & Clause ident (probably only useful internally) \\
\texttt{pd} & Depth of the derivation graph for this clause \\
\texttt{pl} & Number of nodes in the derivation grap \\
\texttt{sc} & Symbol count (function symbols and variables) \\
\texttt{cd} & Depth of the deepest term in the clause \\
\texttt{nl} & Number of literals in the clause \\
\texttt{no} & Number of variable occurences \\
\texttt{nv} & Number of different variables \\
\end{tabular}

\clearpage
\begin{appendix}
  \section{License}
  
  The standard distribution of E is free software. You can use, modify
  and copy it under the terms of the GNU General Public License
  (version 2.0 or later) or the GNU Lesser General Public License
  (version 2.1 or later). You may also have bought a commercial
  version of E from Safelogic A.B. in Gothenburg, Sweden. In this
  case, you are bound by whatever license you agreed to. If you are in
  doubt about which version of E you have, run \texttt{eprover -V} or
  \texttt{eprover -h}\index{GPL}\index{GNU General Public
    License}\index{LGPL}\index{GNU Lesser General Public License}.

  See the file COPYING in the main directory for the full text of the
  licenses. 
\end{appendix}

%\bibliography{stsbib}
\newcommand{\etalchar}[1]{$^{#1}$}
\begin{thebibliography}{WAB{\etalchar{+}}99}

\bibitem[Bac98]{Bachmair:personal-98}
L.~Bachmair.
\newblock {Personal communication at CADE-15, Lindau}.
\newblock Unpublished, 1998.

\bibitem[BDP89]{BDP89}
L.~Bachmair, N.~Dershowitz, and D.A. Plaisted.
\newblock {Completion Without Failure}.
\newblock In H.~Ait-Kaci and M.~Nivat, editors, {\em Resolution of Equations in
  Algebraic Structures}, volume~2, pages 1--30. Academic Press, 1989.

\bibitem[BG94]{BG94}
L.~Bachmair and H.~Ganzinger.
\newblock {Rewrite-Based Equational Theorem Proving with Selection and
  Simplification}.
\newblock {\em Journal of Logic and Computation}, 3(4):217--247, 1994.

\bibitem[CL73]{CL73}
C.~Chang and R.C. Lee.
\newblock {\em {Symbolic Logic and Mechanical Theorem Proving}}.
\newblock Computer Science and Applied Mathematics. Academic Press, 1973.

\bibitem[Der91]{Dershowitz:IJCAI-91}
N.~Dershowitz.
\newblock {Ordering-Based Strategies for Horn Clauses}.
\newblock In J.~Mylopoulos, editor, {\em Proc. of the 12th IJCAI, Sydney},
  volume~1, pages 118--124. Morgan Kaufmann, 1991.

\bibitem[DKS97]{DKS97}
J.~Denzinger, M.~Kronenburg, and S.~Schulz.
\newblock {DISCOUNT: A Distributed and Learning Equational Prover}.
\newblock {\em Journal of Automated Reasoning}, 18(2):189--198, 1997.
\newblock Special Issue on the CADE 13 ATP System Competition.

\bibitem[DS94a]{DS94a}
J.~Denzinger and S.~Schulz.
\newblock {Analysis and Representation of Equational Proofs Generated by a
  Distributed Completion Based Proof System}.
\newblock Seki-Report SR-94-05, Universit{\"a}t Kai\-sers\-lau\-tern, 1994.

\bibitem[DS94b]{DS94b}
J.~Denzinger and S.~Schulz.
\newblock {Recording, Analyzing and Presenting Distributed Deduction
  Processes}.
\newblock In H.~Hong, editor, {\em Proc.\ 1st PASCO, Hagenberg/Linz}, volume~5
  of {\em Lecture Notes Series in Computing}, pages 114--123, Singapore, 1994.
  World Scientific Publishing.

\bibitem[DS96]{DS96a}
J.~Denzinger and S.~Schulz.
\newblock {Recording and Analysing Knowledge-Based Distributed Deduction
  Processes}.
\newblock {\em Journal of Symbolic Computation}, 21(4/5):523--541, 1996.

\bibitem[HBF96]{BHF96}
Th. Hillenbrand, A.~Buch, and R.~Fettig.
\newblock {On Gaining Efficiency in Completion-Based Theorem Proving}.
\newblock In H.~Ganzinger, editor, {\em Proc.\ of the 7th RTA, New Brunswick},
  volume 1103 of {\em LNCS}, pages 432--435. Springer, 1996.

\bibitem[HJL99]{HJL:CADE-99}
Th. Hillenbrand, A.~Jaeger, and B.~L{\"o}chner.
\newblock {System Abstract: Waldmeister -- Improvements in Performance and Ease
  of Use}.
\newblock In H.~Ganzinger, editor, {\em Proc.\ of the 16th CADE, Trento},
  volume 1632 of {\em LNAI}, pages 232--236. Springer, 1999.

\bibitem[McC94]{Mc94}
W.W. McCune.
\newblock {Otter 3.0 Reference Manual and Guide}.
\newblock Technical Report ANL-94/6, Argonne National Laboratory, 1994.

\bibitem[MIL{\etalchar{+}}97]{MILSGSM:JAR-97}
M.~Moser, O.~Ibens, R.~Letz, J.~Steinbach, C.~Goller, J.~Schumann, and K.~Mayr.
\newblock {SETHEO and E-SETHEO -- The CADE-13 Systems}.
\newblock {\em Journal of Automated Reasoning}, 18(2):237--246, 1997.
\newblock Special Issue on the CADE 13 ATP System Competition.

\bibitem[MW97]{MW:JAR-97}
W.W. McCune and L.~Wos.
\newblock {Otter: The CADE-13 Competition Incarnations}.
\newblock {\em Journal of Automated Reasoning}, 18(2):211--220, 1997.
\newblock Special Issue on the CADE 13 ATP System Competition.

\bibitem[NN93]{NN:RTA-93}
P.~Nivela and R.~Nieuwenhuis.
\newblock {Saturation of First-Order (Constrained) Clauses with the Saturate
  System}.
\newblock In C.~Kirchner, editor, {\em Proc.\ of the 5th RTA, Montreal}, volume
  690 of {\em LNCS}, pages 436--440. Springer, 1993.

\bibitem[RV01]{RV:IJCAR-2001}
A.~Riazanov and A.~Voronkov.
\newblock {Vampire 1.1 (System Description)}.
\newblock In R.~Gor{\'e}, A.~Leitsch, and T.~Nipkow, editors, {\em Proc.\ of
  the 1st IJCAR, Siena}, volume 2083 of {\em LNAI}, pages 376--380. Springer,
  2001.

\bibitem[RV02]{RV:AICOM-2002}
A.~Riazanov and A.~Voronkov.
\newblock {The Design and Implementation of VAMPIRE}.
\newblock {\em Journal of AI Communications}, 15(2/3):91--110, 2002.

\bibitem[Sch99]{E:WWW-99}
S.~Schulz.
\newblock {The E Web Site}.
\newblock
  \url{http://www4.informatik.tu-muenchen.de/\-$\sim$schulz/\-WORK/\-eprover.h%
tml}, 1999.

\bibitem[Sch00]{Schulz:Diss-2000}
S.~Schulz.
\newblock {\em {Learning Search Control Knowledge for Equational Deduction}}.
\newblock Number 230 in DISKI. Akademische Verlagsgesellschaft Aka GmbH Berlin,
  2000.
\newblock Ph.D.~Thesis, Fakult{\"a}t f{\"u}r Informatik, Technische
  Universit{\"a}t M{\"u}nchen.

\bibitem[Sch01]{Schulz:KI-2001}
S.~Schulz.
\newblock {Learning Search Control Knowledge for Equational Theorem Proving}.
\newblock In F.~Baader, G.~Brewka, and T.~Eiter, editors, {\em Proc.\ of the
  Joint German/Austrian Conference on Artificial Intelligence (KI-2001)},
  volume 2174 of {\em LNAI}, pages 320--334. Springer, 2001.

\bibitem[Sch02]{Schulz:AICOM-2002}
S.~Schulz.
\newblock {E -- A Brainiac Theorem Prover}.
\newblock {\em Journal of AI Communications}, 15(2/3):111--126, 2002.

\bibitem[Sch04]{Schulz:IJCAR-2004}
S.~Schulz.
\newblock {System Description: E~0.81}.
\newblock In D.~Basin and M.~Rusinowitch, editors, {\em Proc.\ of the 2nd
  IJCAR, Cork, Ireland}, volume 3097 of {\em LNAI}, pages 223--228. Springer,
  2004.

\bibitem[SSCG06]{GSCG:IJCAR-2006}
Geoff Sutcliffe, Stephan Schulz, Koen Claessen, and Allen~Van Gelder.
\newblock {Using the TPTP Language for Writing Derivations and Finite
  Interpretations }.
\newblock In Ulrich Fuhrbach and Natarajan Shankar, editors, {\em Proc.\ of the
  3rd IJCAR, Seattle}, volume 4130 of {\em LNAI}, pages 67--81, 4130, 2006.
  Springer.

\bibitem[Sut05]{Sutcliffe:TPTP-WWW}
G.~Sutcliffe.
\newblock {The TPTP Web Site}.
\newblock \url{http://www.tptp.org}, 2004--2005.

\bibitem[Tam97]{Tammet:JAR-97}
T.~Tammet.
\newblock {Gandalf}.
\newblock {\em Journal of Automated Reasoning}, 18(2):199--204, 1997.
\newblock Special Issue on the CADE 13 ATP System Competition.

\bibitem[WAB{\etalchar{+}}99]{WABCEKTT:CADE-99}
C.~Weidenbach, B.~Afshordel, U.~Brahm, C.~Cohrs, T.~Engel, G.~Jung, E.~Keen,
  C.~Theobalt, and D.~Topic.
\newblock {System Abstract: SPASS Version 1.0.0}.
\newblock In H.~Ganzinger, editor, {\em Proc.\ of the 16th CADE, Trento},
  volume 1632 of {\em LNAI}, pages 378--382. Springer, 1999.

\bibitem[Wei99]{Weidenbach:personal-99}
C.~Weidenbach.
\newblock {Personal communication at CADE-16, Trento}.
\newblock Unpublished, 1999.

\bibitem[Wei01]{Weidenbach:SPASS-2001}
C.~Weidenbach.
\newblock {SPASS: Combining Superposition, Sorts and Splitting}.
\newblock In A.~Robinson and A.~Voronkov, editors, {\em Handbook of Automated
  Reasoning}, volume~II, chapter~27, pages 1965--2013. Elsevier Science and MIT
  Press, 2001.

\bibitem[WGR96]{WGR96}
C.~Weidenbach, B.~Gaede, and G.~Rock.
\newblock {SPASS \& FLOTTER Version 0.42}.
\newblock In M.A. McRobbie and J.K. Slaney, editors, {\em Proc.\ of the 13th
  CADE, New Brunswick}, volume 1104 of {\em LNAI}, pages 141--145. Springer,
  1996.

\end{thebibliography}

\printindex

\end{document}

